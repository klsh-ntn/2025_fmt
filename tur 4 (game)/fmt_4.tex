% arara: xelatex
\documentclass[11pt]{article} % размер шрифта
\usepackage{libertine}

\usepackage{tikz} % картинки в tikz
\usepackage{microtype} % свешивание пунктуации

\usepackage{array} % для столбцов фиксированной ширины

\usepackage{indentfirst} % отступ в первом параграфе

\usepackage{multicol} % текст в несколько колонок
\usepackage{verbatim}

\graphicspath{{images/}} % путь к картинкам

\usepackage{sectsty} % для центрирования названий частей
\allsectionsfont{\centering} % приказываем центрировать все sections

\usepackage{amsmath, amssymb} % куча стандартных математических плюшек

\usepackage[top=1cm, left=1cm, right=1cm, bottom=1cm]{geometry} % размер текста на странице

\usepackage{lastpage} % чтобы узнать номер последней страницы

\usepackage{enumitem} % дополнительные плюшки для списков
%  например \begin{enumerate}[resume] позволяет продолжить нумерацию в новом списке
\usepackage{caption} % подписи к картинкам без плавающего окружения figure

\usepackage{circuitikz} % для рисовки электрических цепей

% \usepackage{physics} % do not use it!

\usepackage{fancyhdr} % весёлые колонтитулы
\pagestyle{empty}
%\lhead{ФМТ}
%\chead{}
%\rhead{КЛШ-2019}
%\lfoot{}
%\cfoot{}
%\rfoot{\thepage/\pageref{LastPage}}
%\renewcommand{\headrulewidth}{0.4pt}
%\renewcommand{\footrulewidth}{0.4pt}



\usepackage{todonotes} % для вставки в документ заметок о том, что осталось сделать
% \todo{Здесь надо коэффициенты исправить}
% \missingfigure{Здесь будет картина Последний день Помпеи}
% команда \listoftodos — печатает все поставленные \todo'шки

\usepackage{booktabs} % красивые таблицы
% заповеди из документации:
% 1. Не используйте вертикальные линии
% 2. Не используйте двойные линии
% 3. Единицы измерения помещайте в шапку таблицы
% 4. Не сокращайте .1 вместо 0.1
% 5. Повторяющееся значение повторяйте, а не говорите "то же"

\usepackage{fontspec} % поддержка разных шрифтов
\usepackage{polyglossia} % поддержка разных языков

\setmainlanguage{russian}
\setotherlanguages{english}

\setmainfont{UbuntuMono Nerd Font}
\usepackage{unicode-math}
\setmathfont{Fira Math}

% \setmainfont{Linux Libertine O} % выбираем шрифт
% можно также попробовать Helvetica, Arial, Cambria и т.Д.
% перешли на \usepackage{libertine}

% чтобы использовать шрифт Linux Libertine на личном компе,
% его надо предварительно скачать по ссылке
% http://www.linuxlibertine.org/

% \newfontfamily{\cyrillicfonttt}{Linux Libertine O}
% пояснение зачем нужно шаманство с \newfontfamily
% http://tex.stackexchange.com/questions/91507/

\AddEnumerateCounter{\asbuk}{\russian@alph}{щ} % для списков с русскими буквами
\setlist[enumerate, 2]{label=\asbuk*),ref=\asbuk*} % списки уровня 2 будут буквами а) б) ...


% делаем короче интервал в списках
\setlength{\itemsep}{0pt}
\setlength{\parskip}{0pt}
\setlength{\parsep}{0pt}


\newcommand{\shortname}{ФМТ: тур 3}

\begin{document}


\section*{Правила игры уголки}
\begin{enumerate}
    \item Один игрок/игрица играет кружочками, второй — треугольниками. 
    \item Игроки/игрицы ходят по очереди, начинает тот, кто выиграл в камень-ножницы-бумага. 
    \item Поле стандартное — 8 на 8 в клетку. 
    \item За один ход игрок/игрица выставляет одну свою фигуру на поле. 
    \item Размещать свои фигуры можно только в свободные клетки.
    \item Если ты поставил 4 фигуры таким образом, что они расположены в углах прямоугольника, то на следующем ходе ты можешь занять его.
    \item Прямоугольник может быть повернутым на \textbf{произвольный угол} относительно границ игрового поля!!!
    \item Ход захвата прямоугольника: \begin{enumerate}
        \item Закрашиваются фигуры в углу прямоугольника
        \item На месте пустых клеток ставятся фигуры, которые принадлежат игроку.
        \item  Можно захватить прямоугольник, даже если внутри его есть чужые фигуры или нет пустых клеток.
    \end{enumerate}
    \item Когда все поле замостится фигурами, у каждого игрока есть право захватить еще по одному прямоугольнику.
    \item Финальное число очков - это число закрашенных фигур игрока.
    \item Побеждает тот, у кого больше закрашенных фигур.
\end{enumerate}

\begin{figure}[h!]
\begin{minipage}{0.5\textwidth}
    \centering{\includegraphics[scale=0.5]{1.png}}
    \caption{Ход захвата прямоугольника}
\end{minipage}
\begin{minipage}{0.5\textwidth}
    \centering{\includegraphics[scale=0.5]{2.png}}
    \caption{Ход захвата прямоугольника с поворотом}
\end{minipage}
\end{figure}

Счет в игре - это разница очков. Если по итогам всех раундов у игрока/игрицы положительная сумма разниц, то он победил - 2 очка. Если сумма равна $0$, то они сыграли в ничьюa - $1$ очко. Если же сумма $<0$, то ставится $0$ очков - поражение.

\newpage
\section*{Организация тура}

Игровой тур состоит из четырёх фаз: объяснение правил, сыгрывание, индивидуальные партии, командные партии.

\begin{enumerate}
    \item Во время сыгрывания команда и вольные стрелки объединяются и играют тренировочные партии между собой. 
    Судьи отвечают на вопросы игроков. На объяснение правил и сыгрывание команд отводится примерно 20 минут. 
    \item После сыгрывания капитан каждой команды объявляет основной состав команды из четырёх человек и вольных стрелков в любом количестве.
    \item Во время индивидуальных партий основной состав команд А и Б делятся на пары: АБ, АБ, АБ, АБ. 
    \item Вольные стрелки играют как единый дополнительный пятый игрок своей команды. 
    Вольные стрелки играют против вольных стрелков. 
    \item За индивидуальный этап каждый игрок команды А и каждый игрок команды Б играет одну партию крестиками и одну партию ноликами. 
    \item За индивидуальный этап команда получает сумму очков, набранных всеми её игроками. 
    \item На индивидуальный этап отводится примерно 15 минут\footnote{Если время индивидуального этапа позволяет и оба игрока в паре и судьи согласны, то игроки могут сыграть ещё пару партий (А — крестиками, Б — ноликами и наоборот)}.
    \item На командном этапе команда вместе с вольными стрелками играет как единое целое. 
    \item Ход выполняет капитан команды у доски. На обсуждение каждого хода отводится 40 секунд. 
    \item В командном этапе каждая команда играет одну партию крестиками и одну партию ноликами.
    \item За командный этап команда получает утроенное количество набранных очков. 
    \item Итоговые очки за игровой тур равны сумме очков за индивидуальный и командный этапы.
    \item Судьи отмечают в протоколе имена игроков основного состава в индивидуальном этапе, очки, которые они заработали, 
    а также очки, набранные вольными стрелками, и очки в командном этапе. 
\end{enumerate}






\end{document}

\newcommand{\judgenormal}{\textbf{TOP SECRET!!! Судейский экземпляр!!!} \hspace*{1cm} \textbf{\shortname} \hspace*{1cm} \textbf{Обычные столы}}
\newcommand{\judgetop}{\textbf{TOP SECRET!!! Судейский экземпляр!!!} \hspace*{1cm} \textbf{\shortname} \hspace*{1cm} \textbf{top-3 столы}}

\newcommand{\putlogo}{
\begin{center}
\begin{tabular}{cc}
\includegraphics[scale=0.05]{../klsh_logos/klsh_logo_2.png} &
\raisebox{1cm}{
    {\Large\bf \shortname \hspace*{8cm} КЛШ $7^2 - 2$}
}
\end{tabular}
\end{center}
}

\newcommand{\judgenotes}{
    За одну итерацию оппонирования можно получить максимум 1 балл. 
    Вольные стрелки приносят команде от 0 до 3 баллов. 
    Штрафы за выход за три минуты при решении своей задачи: от 0 до 30 секунд — 1 балл штрафа, 
    от 30 до 60 секунд — 2 балла штрафа и далее 3 балла штрафа.
}


\newcommand{\problemA}{
В КЛШ-47 в обращение ввели монеты достоинством в $1$, $2$, $3$, \dots, $19$, $20$ лапок. 
У Арины Медведевой была одна монета. 
Она купила шоколадку и получила одну монету сдачи. 
Снова купила такую же шоколадку и получила сдачу тремя разными монетами. 
Хотела купить третью шоколадку, но денег не хватило. 

Сколько лапок стоит шоколадка?
}


\newcommand{\solutionA}{
Обозначим стоимость шоколадки за $A$. 
На вторую покупку сдачу $D$ выдали тремя различными монетами, $1 + 2 + 3 = 6 \leq D$. 
Стоимость обязательно больше сдачи, $6 + 1 = 7 \leq D + 1 \leq A$. 
Следовательно, общие затраты, $S = 2A + D \geq 2 \cdot 7 + 6 = 20$. 
С другой стороны, $S \leq 20$, первоначально Арина пришла с одной лишь монетой, ценность которой не превышает $20$. 

Отсюда, $S = 20$, $D = 6$, $A = 7$.  

Полный перебор: 3 балла. Угаданный ответ без аргументации и полного перебора: 1 балл. 
}


\newcommand{\problemB}{
В треугольник вписана окружность с радуисом $4$. 
Точка касания окружности делит одну из сторон треугольника на кусочки $8$ и $6$ см. 
Найди сумму двух других сторон треугольника. 
}
\newcommand{\solutionB}{
Обозначим $x$ неизвестный кусочек на двух сторонах треугольника. 
Считаем площадь двумя способами:
\[
 \sqrt{(4 + x)\cdot x \cdot 6 \cdot 8}= S = \frac{8 + x + x + 6 + 8 + 6}{2} \cdot 4.
\]
Отсюда $x = 7$ и искомая сумма равна $28$.
}

% \newcommand{\hardmath}{
% Двадцать четыре команды КЛШ-2023 поделили между собой все натуральные числа. 
% Оказалось, что числа доставшиеся каждой команде — это бесконечная арифметическая прогрессия. 

% Найди сумму величин, обратных разностям этих арифметических прогрессий. 
% }
% \newcommand{\solhardmath}{
% Возьмём большое $n$ и посмотрим, какую долю этого ряда получила команда с разностью $d$.
% Замечаем, что выбрав достаточно большое $n$, можно сколь угодно приблизить долю к $1/d$. 

% Все натуральные числа полностью поделены между командами, 
% поэтому сумма $1/d_{\alpha} + \ldots + 1/d_{\omega} = 1$. 

% Угадан ответ 1 на примере частного случая: 0 баллов и переход. 
% }


\newcommand{\problemC}{
У Вани Сапогова глаза находятся на высоте $h$. 
Какова должна быть минимальная высота вертикального зеркала, 
чтобы Ваня Сапогов мог видеть в нём себя от кончиков сапогов до глаз? 

Зондера готовы повесить зеркало на любой необходимой высоте :)}
\newcommand{\solutionC}{
Треугольники: глаза-настоящее зеркало и глаза-изображение Вани подобны с коэффициентом $2$. 
Ответ: $h/2$
}

\newcommand{\problemD}{
Тима Спрыжков собрал необычную конструкцию.
Он прикрепил палку массой $M$ к стене на шарнир в точке $C$.
Нить соединяет точку $A$ на стене и конец палки $B$.
Угол $\alpha$ известен.

\includegraphics[width=.145\textwidth]{Tur_3easy.png}

Определи силу натяжения нити $AB$. 
}

\newcommand{\solutionD}{
Правило моментов относительно точки $C$:
\[
Mg \frac{\ell}{2} \cos \alpha = T\ell \sin \alpha.
\]
Отсюда $T = Mg / (2\tg \alpha)$. 
}





\newcommand{\problemAA}{
\problemA
}
\newcommand{\solutionAA}{
\solutionA
}




\newcommand{\problemBB}{
Серёже Ламзину у отвала приснилась трапеция с основаниями $4$ и $11$ см и диагоналями $9$ и $12$ см. 

Чему равна площадь трапеции?
}

\newcommand{\solutionBB}{
Решение 1. Отложим основание $4$ см правее основная в $11$ см. 
Узнаём прямоугольный треугольник со сторонами $9$, $12$ и $15$. 
Следовательно, угол между диагоналями прямой и площадь равна $S = 0.5 \cdot 9 \cdot 12 = 54$. 

Решение 2. Обозначим высоту буквой $h$, проецируем короткое основание на длинное. 
Два раза используем теорему Пифагора и складываем нижнюю сторону из кусочков:
\[
\sqrt{9^2 - h^2} + \sqrt{12^2 - h^2} = 11 + 4.
\]
Решаем уравнение $h = 72/7$ и находим площадь $S = 54$.
}


\newcommand{\problemCC}{
Два зеркала образуют двугранный угол $\varphi$. 
На одно из зеркал падает луч под углом $\alpha$ к перпендикуляру к зеркалу.
На какой угол отклонится этот луч после двух отражений?

Все лучи перпендикулярны ребру угла, $\alpha < \varphi < \pi/2$. 
}

\newcommand{\solutionCC}{
Углы в треугольнике «зеркало-зеркало-луч» равны $\varphi$, $\pi/2 - \alpha$, $\pi/2 - \varphi + \alpha$.
Углы в треугольнике из лучей равны $2\alpha$, $2\phi - 2\alpha$, $\pi - 2\phi$.

Ответ: $\pi - 2\phi$. 
}




\newcommand{\problemDD}{
Тима Спрыжков собрал необычную конструкцию.
Он прикрепил палку массой $M$ к стене на шарнир в точке $C$.
Первая нить соединяет точку $A$ на стене и конец палки $B$,
вторая нить — конец палки $B$ и груз массы $m$.
Угол $\alpha$ известен.

\includegraphics[width=.145\textwidth]{Tur_3.png}

Определи силу натяжения нити $AB$. 
}

\newcommand{\solutionDD}{
Правило моментов относительно точки $C$:
\[
Mg \frac{\ell}{2} \cos \alpha + mg \ell \cos \alpha - T\ell \sin \alpha = 0.
\]
Отсюда $T = (M/2 + m) g / \tg \alpha$. 
}


\newcommand{\judgenormalpage}{
\judgenormal

\begin{small}
    \judgenotes
\end{small}

\begin{enumerate}
    \item \problemA
    
    \solutionA
    \item \problemB
    
    \solutionB
    \item \problemC
    
    \solutionC
    \item \problemD
    
    \solutionD
\end{enumerate}
}

\newcommand{\judgetoppage}{
\judgetop

\begin{small}
    \judgenotes
\end{small}

\begin{enumerate}
    \item \problemAA
    
    \solutionAA
    \item \problemBB
    
    \solutionBB
    \item \problemCC
    
    \solutionCC
    \item \problemDD
    
    \solutionDD
\end{enumerate}
}


\newcommand{\studentnormalpage}{
\putlogo

\begin{enumerate}
    \item \problemA
    \item \problemB
    \item \problemC
    \item \problemD
\end{enumerate}
}

\newcommand{\studenttoppage}{
\putlogo

\begin{enumerate}
    \item \problemAA
    \item \problemBB
    \item \problemCC
    \item \problemDD
\end{enumerate}        
}


\begin{document}

% \newpage
\judgenormalpage

\newpage
\judgenormalpage


\newpage % школьный экземпляр обычных столов
\studentnormalpage

\vfill

\studentnormalpage

% \end{document} %% если деления задач по столам нет


\newpage 
\judgetoppage

\newpage 
\judgetoppage


\newpage %% школьный экземпляр top3 столов
\studenttoppage

\vfill

\studenttoppage


\end{document}



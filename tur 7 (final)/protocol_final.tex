% arara: pdflatex
\documentclass[12pt]{article}


% если не менять дизайн, то поправить надо только 
% \newcommand{\tour}{...}
% и 
% \newcommand{\blockofjudges}{...}

\newcommand{\tour}{Финал ФМТ} % номер тура

\newcommand{\blockofjudges}{

% table x
% \protocol{where}{team-a}{team-b}{top judge}{more judges} % comment


% все сотрудники:
% Ян Шапиро, Ольга Володько, Вова Федоров, Миша Красков, Дмитрий Федченко, Лев Назаров,
% Тима Спрыжков, Паша Рябенко, Марина Хмельницкая, Никита Терентьев, Вова Носков,
% Антон Шейкин, Саша Тимошков, Настя Судницына, Миша Торшин, Егор Лунев
% Егор Скурковин, Вика Фонарева, Ваня Адо, Влада Синицына, Маша Казаринова
% Сережа Ламзин, Денис Гохфельд, Герман Злобин, Рома Лисин, Андрей Трегубович,
% Роберт Гринштейн, Саша Акантьев, Иван Карлов


% пустой протокол для ручного заполнения
% пустой протокол для ручного заполнения
% day 0
\protocol{2.2}{$\eta$}{$\tau$}{Рома Лисин}{Роберт Гринштейн, Ваня Адо, Ян Шапиро, Егор Лунев, Лев Назаров}

\protocol{3.2}{$\nu$}{$\alpha$}{Вова Федоров}{Боря Демешев, Сережа Ламзин, Андрей Трегубович, Марина Хмельницкая, Вова Носков}
}

\usepackage[utf8]{inputenc}
\usepackage[russian]{babel}
\usepackage{wrapfig}
\usepackage{amsmath}
\usepackage{amssymb}
\usepackage{geometry}
\usepackage{graphicx}
\usepackage{rotating}
\usepackage{makecell}
\usepackage{array}
\usepackage{tikzsymbols}
\newcolumntype{P}[1]{>{\centering\arraybackslash}p{#1}}
\newcolumntype{M}[1]{>{\centering\arraybackslash}m{#1}}

\geometry{top=0.5cm}
\geometry{bottom=0.5cm}
\geometry{left=1cm}
\geometry{right=1cm}
\pagestyle{empty}

\usepackage{booktabs}
% заповеди из документации:
% 1. Не используйте вертикальные линни
% 2. Не используйте двойные линии
% 3. Единицы измерения - в шапку таблицы
% 4. Не сокращайте .1 вместо 0.1
% 5. Повторяющееся значение повторяйте, а не говорите "то же"

\newcommand{\myrotcell}[1]{\rotcell{\makebox[0pt][l]{#1}}}


\newcommand{\protocol}[5]{
\begin{minipage}{.2\textwidth}
    \begin{center}
    \includegraphics[width=0.48\textwidth]{klsh_logo.pdf}
    \end{center}
\end{minipage}
\begin{minipage}{.6\textwidth}
    \begin{flushleft}
    \textbf{Старший судья}: #4\\ 
    \textbf{Судьи}: #5 \\ 
    \textbf{Аудитория}: #1 \\ 
    \end{flushleft}
\end{minipage}
\begin{minipage}{.2\textwidth}
    \begin{flushleft}
        {\Large\textbf{\tour}} \\
        \textbf{Команды}: #2 --- #3 
    \end{flushleft}
\end{minipage}


\vspace*{0.2cm}
\begin{tabular}{ | M{0.4cm} | M{2cm} | M{6.1cm} | M{0.8cm} | M{0.8cm} | M{0.5cm} | M{6.1cm} |}
    \hline
    \Smiley[1.6] & Время & Ответ и игрок #2 & балл #2 & балл #3 & \tiny{ $\times$ $\leftarrow$ $\rightarrow$} & Ответ и игрок #3 \\
    \hline
    \rotatebox{90}{Задача 1\hspace*{1cm}} & & & & & & \\ [2.1cm]
    \hline
    \rotatebox{90}{Задача 2\hspace*{1cm}} & & & & & & \\ [2.1cm]
    \hline
    \rotatebox{90}{Задача 3\hspace*{1cm}} & & & & & & \\ [2.1cm]
    \hline
    \rotatebox{90}{Задача 4\hspace*{1cm}} & & & & & & \\ [2.1cm]
    \hline
\end{tabular}

\vspace*{0.25cm} 

\begin{minipage}{.42\textwidth}

%Условные значки: \\
$\times$ задача снята \\
$\rightarrow$ переход от #2 к #3 \\
$\leftarrow$ переход от #3 к #2 \\

\begin{tabular}{ | m{5cm} | M{1cm} | M{1cm} |}
\hline
\textbf{} & #2 & #3 \\
\hline
Итог основного этапа & & \\ [0.8cm]
\hline 
\end{tabular}

\vspace*{1cm}
\textbf{Подписи команд:}\dotfill\\

\vspace*{0.5cm}
\textbf{Подписи судей:}\dotfill\\
\end{minipage}
\hspace*{0.5cm}
\begin{minipage}{.55\textwidth}
\begin{tabular}{ | m{5cm} | M{2.3cm} | M{2.3cm} |}
\hline
\textbf{Этап обмена ударами} & #2 & #3 \\
\hline 
\vspace*{0.4cm}Решение задачи соперника\vspace*{1cm} Докладчик & & \\ [1.6cm]
\hline
Штраф за незнание & & \\ [0.6cm]
\hline
Дисциплинарный штраф & & \\ [0.6cm]
\hline
Красота обменной задачи & & \\ [0.6cm]
\hline
\textbf{Итоговый счёт} & & \\ [0.6cm]
\hline
\end{tabular}
\end{minipage}
\newpage
}
\end{document}
